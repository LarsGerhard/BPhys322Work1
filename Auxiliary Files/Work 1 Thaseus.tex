\documentclass[11pt]{article}

\usepackage{amsmath,setspace,mathtools,amssymb,booktabs,graphicx, multicol}

\usepackage[utf8]{inputenc}

\usepackage[letterpaper,portrait,margin=0.5cm]{geometry}

\graphicspath{ {./} }


\title{Work 1 }

\author{Thaseus Karkabe-Olson}

\date{}

\begin{document}
	
	
	\maketitle
		\raggedright

		\section*{Problem 5.2}
			From example 5.2 we know that our equations of motion are
	
			\begin{gather*}
			    y(t) = C_1 Cos(\omega t) + C_2 Sin(\omega t) + \frac{E}{B}t + C_3\\
			    z(t) = C_2 Cos(\omega t) - C1 Sin(\omega t) + C_4\\
			\end{gather*}
			
			where
			
			\[ \omega = \frac{Q B}{m} \]
				
			\subsection*{Part b}
			
				We then know
				
				\[\b{x}_0 = 0 \text{ and }\b{v}_0 = \frac{E}{2B}\hat{y} \]
				

				We can put in our initial conditions to get
				
				\begin{gather*}
				    0 = C_1 + C_3\\
				    0 = C_2 + C_4\\
				    \frac{E}{2B} = C_2 \omega + \frac{E}{B}\\
				    0 = -C_1 \omega \\
				\end{gather*}
				
				Back to my favourite program to find the constants
				
				\[\text{Input: } \text{FullSimplify}\left[\text{RowReduce}\left[\left(
				\begin{array}{ccccc}
					\cos (0) & \sin (0) & 1 & 0 & 0 \\
					-\sin (0) & \cos (0) & 0 & 1 & 0 \\
					-\omega  \sin (0) & \omega  \cos (0) & 0 & 0 & \frac{e}{2 B} \\
					-\omega  \cos (0) & -\omega  \sin (0) & 0 & 0 & 0 \\
				\end{array}
				\right)\right]\right] \]
				
				\[\text{Output: } \left(
				\begin{array}{ccccc}
					1 & 0 & 0 & 0 & 0 \\
					0 & 1 & 0 & 0 & \frac{e}{2 B \omega } \\
					0 & 0 & 1 & 0 & 0 \\
					0 & 0 & 0 & 1 & -\frac{e}{2 B \omega } \\
				\end{array}
				\right) \]
				
				After simplifying, it turns out that our equations of motion are
				
				\begin{gather*}
				    y(t) = \frac{E (2 t \omega +\sin (t \omega ))}{2 B \omega }\\
				    z(t) = \frac{E (\cos (t \omega )-1)}{2 B \omega }\\
				\end{gather*}

				Using some sample numbers and running these equations in python, we get the following behaviour

				\includegraphics[scale = 0.75]{Figure_1}

				From this we can conclude that there is no motion in the x direction, that the particle follows an almost linear path in the y direction, and that it spins in a circle in the z direction

				
			
			\subsection*{Part c}

				We then know

				\[\b{x}_0 = 0 \text{ and }\b{v}_0 = \frac{E}{B}(\hat{y} + \hat{z}) \]


				We can put in our initial conditions to get

				\begin{gather*}
				    0 = C_1 Cos(\omega t) + C_2 Sin(\omega t) + \frac{E}{B}t + C_3\\
				    0 = C_2 Cos(\omega t) - C1 Sin(\omega t) + C_4\\
				    \frac{E}{B} = - C_1 \omega Sin(\omega t) + C_2 \omega Cos(\omega t) + \frac{E}{B}\\
				    \frac{E}{B} = -C_1 \omega Cos(\omega t) - C_2 \omega Sin(\omega t)\\
				\end{gather*}




\end{document}
